\documentclass[12pt]{article}

\usepackage{color}
\usepackage[utf8]{inputenc}
\usepackage{fullpage}
\usepackage{indentfirst}
\usepackage{graphicx}
\usepackage{graphicx}
\usepackage{amssymb, amsmath}
\usepackage[frenchb]{babel}



\usepackage[pdftex=true,
hyperindex=true,
colorlinks=true]{hyperref}



\newsavebox{\boiteremarque}
\newenvironment{remarque}{%
% clause begin
\begin{lrbox}{\boiteremarque}% début mise en boîte
\begin{minipage}{.8\textwidth}}{%
% clause end
\end{minipage}
\end{lrbox}% fin mise en boîte
% production de la boîte encadrée
\begin{center}
\fbox{\usebox{\boiteremarque}}
\end{center}}




\title{Génératrice portable \\ Construction d'une alimentation \\ stabilisée en tension portable}
\author{Jérôme GRARD}
\addtolength{\textwidth}{1cm}
\addtolength{\oddsidemargin}{-0,5cm}


\begin{document}


\maketitle
\thispagestyle{empty}

\begin{remarque}
	Ce document a pour but de présenter les travaux de recherche permettant de construire une
	alimentation stabilisée en tension portable. Cette alimentation a pour contraintes d'utiliser
	des pièces récupérées en priorité ---pour alléger les coûts--- et de fonctionner avec des
	piles AA (LR6) de sorte à pouvoir les remplacer facilement et à moindre coût.\newline
	\begin{center}
		Ce document a été édité en \LaTeX --- Diffusons le logiciel libre ! \newline
		Mes autres projets sur http://the-destiny.no-ip.org
	\end{center}
\end{remarque}

\newpage



\pagestyle{plain}
\setcounter{page}{1}
\tableofcontents


\newpage


\end{document}
